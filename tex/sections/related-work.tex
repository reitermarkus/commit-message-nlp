\section{Related Work}
\label{sec:related-work}

The idea of classifying commits is based on the concept of labelling
messages with a specific tag. Already labelled commit messages can be
seen as correctly labelled, as the author is unlike to label his commit
incorrectly. Furthermore we have many authors for our selection of
messages essentially giving us a collaborative labelling approach,
which has already been well researched \cite{Golder2006}.

While this study only focuses on the lexicographical analysis of the
commit message, we can also infer a lot of information about the meaning
of a commit through other factors like code changes. From this we can
conclude if a commit only fixes some mistakes or introduces completely
new features by analysing its size \cite{Hindle2008}.

We can also use commit messages to infer information about the
changes themselves, which can be especially useful when doing version
control analysis. Automatically tagging and understanding the idea
or reason behind a commit can reduce the effort of others working
with the same codebase. Correct labelling can reduce the effort
of filtering commits relevant to a certain part of the codebase or
make it easier for developers to find relevant changes by other
contributors \cite{Mockus2000}.

