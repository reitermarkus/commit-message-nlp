\section{Related Work}
\label{sec:related-work}

The idea of classifying commits is based on the concept of labelling
messages with a specific tag. Already labelled commit messages can be
seen as correctly labelled, as the author is unlikely to label his commit
incorrectly. Furthermore we have many authors for our selection of
messages essentially giving us a collaborative labelling approach,
which has already been well researched to result in a representative
data set \cite{Golder2006}.

While this study only focuses on the lexicographical analysis of the
commit message, we can also infer a lot of information about the meaning
of a commit through other factors like code changes. From this we can
conclude if a commit only fixes some mistakes or introduces completely
new features by analysing its size \cite{Hindle2008}.

We can also use commit messages to infer information about the
changes themselves, which can be especially useful when doing version
control analysis. Automatically tagging and understanding the idea
or reason behind a commit can reduce the effort of others working
with the same codebase. Correct labelling can reduce the effort
of filtering commits relevant to a certain part of the codebase or
make it easier for developers to find relevant changes by other
contributors \cite{Mockus2000}. An example would be to check if
a commit introduces a fix for an issue. For some repositories
it was determined that the amount of changes directly contributes
to the likelihood of being a fix, \cite{Sliwerski2005}. However,
this might not be true for all codebases in general.

While gathering information using all metrics of a commit can yield
good results, only analysing the message itself is less trivial.
Using many different repositories for commit resources also greatly
increases diversity, which can ultimately lead to lower accuracy
of a predictive model \cite{Mockus2000,Santos2020}.

The most important factor in commit classification is the selection
of labels, as they can greatly influence the prediction results.
However, applying to many labels might result in a very complex
classification problem, where commits cannot be properly fitted
to a single category ultimately resulting in wrong predictions
\cite{Santos2020}.
